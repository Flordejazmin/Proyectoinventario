% Options for packages loaded elsewhere
\PassOptionsToPackage{unicode}{hyperref}
\PassOptionsToPackage{hyphens}{url}
\PassOptionsToPackage{dvipsnames,svgnames,x11names}{xcolor}
%
\documentclass[
  letterpaper,
  DIV=11,
  numbers=noendperiod]{scrreprt}

\usepackage{amsmath,amssymb}
\usepackage{iftex}
\ifPDFTeX
  \usepackage[T1]{fontenc}
  \usepackage[utf8]{inputenc}
  \usepackage{textcomp} % provide euro and other symbols
\else % if luatex or xetex
  \usepackage{unicode-math}
  \defaultfontfeatures{Scale=MatchLowercase}
  \defaultfontfeatures[\rmfamily]{Ligatures=TeX,Scale=1}
\fi
\usepackage{lmodern}
\ifPDFTeX\else  
    % xetex/luatex font selection
\fi
% Use upquote if available, for straight quotes in verbatim environments
\IfFileExists{upquote.sty}{\usepackage{upquote}}{}
\IfFileExists{microtype.sty}{% use microtype if available
  \usepackage[]{microtype}
  \UseMicrotypeSet[protrusion]{basicmath} % disable protrusion for tt fonts
}{}
\makeatletter
\@ifundefined{KOMAClassName}{% if non-KOMA class
  \IfFileExists{parskip.sty}{%
    \usepackage{parskip}
  }{% else
    \setlength{\parindent}{0pt}
    \setlength{\parskip}{6pt plus 2pt minus 1pt}}
}{% if KOMA class
  \KOMAoptions{parskip=half}}
\makeatother
\usepackage{xcolor}
\setlength{\emergencystretch}{3em} % prevent overfull lines
\setcounter{secnumdepth}{5}
% Make \paragraph and \subparagraph free-standing
\makeatletter
\ifx\paragraph\undefined\else
  \let\oldparagraph\paragraph
  \renewcommand{\paragraph}{
    \@ifstar
      \xxxParagraphStar
      \xxxParagraphNoStar
  }
  \newcommand{\xxxParagraphStar}[1]{\oldparagraph*{#1}\mbox{}}
  \newcommand{\xxxParagraphNoStar}[1]{\oldparagraph{#1}\mbox{}}
\fi
\ifx\subparagraph\undefined\else
  \let\oldsubparagraph\subparagraph
  \renewcommand{\subparagraph}{
    \@ifstar
      \xxxSubParagraphStar
      \xxxSubParagraphNoStar
  }
  \newcommand{\xxxSubParagraphStar}[1]{\oldsubparagraph*{#1}\mbox{}}
  \newcommand{\xxxSubParagraphNoStar}[1]{\oldsubparagraph{#1}\mbox{}}
\fi
\makeatother

\usepackage{color}
\usepackage{fancyvrb}
\newcommand{\VerbBar}{|}
\newcommand{\VERB}{\Verb[commandchars=\\\{\}]}
\DefineVerbatimEnvironment{Highlighting}{Verbatim}{commandchars=\\\{\}}
% Add ',fontsize=\small' for more characters per line
\usepackage{framed}
\definecolor{shadecolor}{RGB}{241,243,245}
\newenvironment{Shaded}{\begin{snugshade}}{\end{snugshade}}
\newcommand{\AlertTok}[1]{\textcolor[rgb]{0.68,0.00,0.00}{#1}}
\newcommand{\AnnotationTok}[1]{\textcolor[rgb]{0.37,0.37,0.37}{#1}}
\newcommand{\AttributeTok}[1]{\textcolor[rgb]{0.40,0.45,0.13}{#1}}
\newcommand{\BaseNTok}[1]{\textcolor[rgb]{0.68,0.00,0.00}{#1}}
\newcommand{\BuiltInTok}[1]{\textcolor[rgb]{0.00,0.23,0.31}{#1}}
\newcommand{\CharTok}[1]{\textcolor[rgb]{0.13,0.47,0.30}{#1}}
\newcommand{\CommentTok}[1]{\textcolor[rgb]{0.37,0.37,0.37}{#1}}
\newcommand{\CommentVarTok}[1]{\textcolor[rgb]{0.37,0.37,0.37}{\textit{#1}}}
\newcommand{\ConstantTok}[1]{\textcolor[rgb]{0.56,0.35,0.01}{#1}}
\newcommand{\ControlFlowTok}[1]{\textcolor[rgb]{0.00,0.23,0.31}{\textbf{#1}}}
\newcommand{\DataTypeTok}[1]{\textcolor[rgb]{0.68,0.00,0.00}{#1}}
\newcommand{\DecValTok}[1]{\textcolor[rgb]{0.68,0.00,0.00}{#1}}
\newcommand{\DocumentationTok}[1]{\textcolor[rgb]{0.37,0.37,0.37}{\textit{#1}}}
\newcommand{\ErrorTok}[1]{\textcolor[rgb]{0.68,0.00,0.00}{#1}}
\newcommand{\ExtensionTok}[1]{\textcolor[rgb]{0.00,0.23,0.31}{#1}}
\newcommand{\FloatTok}[1]{\textcolor[rgb]{0.68,0.00,0.00}{#1}}
\newcommand{\FunctionTok}[1]{\textcolor[rgb]{0.28,0.35,0.67}{#1}}
\newcommand{\ImportTok}[1]{\textcolor[rgb]{0.00,0.46,0.62}{#1}}
\newcommand{\InformationTok}[1]{\textcolor[rgb]{0.37,0.37,0.37}{#1}}
\newcommand{\KeywordTok}[1]{\textcolor[rgb]{0.00,0.23,0.31}{\textbf{#1}}}
\newcommand{\NormalTok}[1]{\textcolor[rgb]{0.00,0.23,0.31}{#1}}
\newcommand{\OperatorTok}[1]{\textcolor[rgb]{0.37,0.37,0.37}{#1}}
\newcommand{\OtherTok}[1]{\textcolor[rgb]{0.00,0.23,0.31}{#1}}
\newcommand{\PreprocessorTok}[1]{\textcolor[rgb]{0.68,0.00,0.00}{#1}}
\newcommand{\RegionMarkerTok}[1]{\textcolor[rgb]{0.00,0.23,0.31}{#1}}
\newcommand{\SpecialCharTok}[1]{\textcolor[rgb]{0.37,0.37,0.37}{#1}}
\newcommand{\SpecialStringTok}[1]{\textcolor[rgb]{0.13,0.47,0.30}{#1}}
\newcommand{\StringTok}[1]{\textcolor[rgb]{0.13,0.47,0.30}{#1}}
\newcommand{\VariableTok}[1]{\textcolor[rgb]{0.07,0.07,0.07}{#1}}
\newcommand{\VerbatimStringTok}[1]{\textcolor[rgb]{0.13,0.47,0.30}{#1}}
\newcommand{\WarningTok}[1]{\textcolor[rgb]{0.37,0.37,0.37}{\textit{#1}}}

\providecommand{\tightlist}{%
  \setlength{\itemsep}{0pt}\setlength{\parskip}{0pt}}\usepackage{longtable,booktabs,array}
\usepackage{calc} % for calculating minipage widths
% Correct order of tables after \paragraph or \subparagraph
\usepackage{etoolbox}
\makeatletter
\patchcmd\longtable{\par}{\if@noskipsec\mbox{}\fi\par}{}{}
\makeatother
% Allow footnotes in longtable head/foot
\IfFileExists{footnotehyper.sty}{\usepackage{footnotehyper}}{\usepackage{footnote}}
\makesavenoteenv{longtable}
\usepackage{graphicx}
\makeatletter
\def\maxwidth{\ifdim\Gin@nat@width>\linewidth\linewidth\else\Gin@nat@width\fi}
\def\maxheight{\ifdim\Gin@nat@height>\textheight\textheight\else\Gin@nat@height\fi}
\makeatother
% Scale images if necessary, so that they will not overflow the page
% margins by default, and it is still possible to overwrite the defaults
% using explicit options in \includegraphics[width, height, ...]{}
\setkeys{Gin}{width=\maxwidth,height=\maxheight,keepaspectratio}
% Set default figure placement to htbp
\makeatletter
\def\fps@figure{htbp}
\makeatother
% definitions for citeproc citations
\NewDocumentCommand\citeproctext{}{}
\NewDocumentCommand\citeproc{mm}{%
  \begingroup\def\citeproctext{#2}\cite{#1}\endgroup}
\makeatletter
 % allow citations to break across lines
 \let\@cite@ofmt\@firstofone
 % avoid brackets around text for \cite:
 \def\@biblabel#1{}
 \def\@cite#1#2{{#1\if@tempswa , #2\fi}}
\makeatother
\newlength{\cslhangindent}
\setlength{\cslhangindent}{1.5em}
\newlength{\csllabelwidth}
\setlength{\csllabelwidth}{3em}
\newenvironment{CSLReferences}[2] % #1 hanging-indent, #2 entry-spacing
 {\begin{list}{}{%
  \setlength{\itemindent}{0pt}
  \setlength{\leftmargin}{0pt}
  \setlength{\parsep}{0pt}
  % turn on hanging indent if param 1 is 1
  \ifodd #1
   \setlength{\leftmargin}{\cslhangindent}
   \setlength{\itemindent}{-1\cslhangindent}
  \fi
  % set entry spacing
  \setlength{\itemsep}{#2\baselineskip}}}
 {\end{list}}
\usepackage{calc}
\newcommand{\CSLBlock}[1]{\hfill\break\parbox[t]{\linewidth}{\strut\ignorespaces#1\strut}}
\newcommand{\CSLLeftMargin}[1]{\parbox[t]{\csllabelwidth}{\strut#1\strut}}
\newcommand{\CSLRightInline}[1]{\parbox[t]{\linewidth - \csllabelwidth}{\strut#1\strut}}
\newcommand{\CSLIndent}[1]{\hspace{\cslhangindent}#1}

\KOMAoption{captions}{tableheading}
\makeatletter
\@ifpackageloaded{bookmark}{}{\usepackage{bookmark}}
\makeatother
\makeatletter
\@ifpackageloaded{caption}{}{\usepackage{caption}}
\AtBeginDocument{%
\ifdefined\contentsname
  \renewcommand*\contentsname{Table of contents}
\else
  \newcommand\contentsname{Table of contents}
\fi
\ifdefined\listfigurename
  \renewcommand*\listfigurename{List of Figures}
\else
  \newcommand\listfigurename{List of Figures}
\fi
\ifdefined\listtablename
  \renewcommand*\listtablename{List of Tables}
\else
  \newcommand\listtablename{List of Tables}
\fi
\ifdefined\figurename
  \renewcommand*\figurename{Figure}
\else
  \newcommand\figurename{Figure}
\fi
\ifdefined\tablename
  \renewcommand*\tablename{Table}
\else
  \newcommand\tablename{Table}
\fi
}
\@ifpackageloaded{float}{}{\usepackage{float}}
\floatstyle{ruled}
\@ifundefined{c@chapter}{\newfloat{codelisting}{h}{lop}}{\newfloat{codelisting}{h}{lop}[chapter]}
\floatname{codelisting}{Listing}
\newcommand*\listoflistings{\listof{codelisting}{List of Listings}}
\makeatother
\makeatletter
\makeatother
\makeatletter
\@ifpackageloaded{caption}{}{\usepackage{caption}}
\@ifpackageloaded{subcaption}{}{\usepackage{subcaption}}
\makeatother

\ifLuaTeX
  \usepackage{selnolig}  % disable illegal ligatures
\fi
\usepackage{bookmark}

\IfFileExists{xurl.sty}{\usepackage{xurl}}{} % add URL line breaks if available
\urlstyle{same} % disable monospaced font for URLs
\hypersetup{
  pdftitle={SISTEMA DE INVENTARIO},
  pdfauthor={Jazmin Sarahí Flores Gómez},
  colorlinks=true,
  linkcolor={blue},
  filecolor={Maroon},
  citecolor={Blue},
  urlcolor={Blue},
  pdfcreator={LaTeX via pandoc}}


\title{SISTEMA DE INVENTARIO}
\author{Jazmin Sarahí Flores Gómez}
\date{Invalid Date}

\begin{document}
\maketitle

\renewcommand*\contentsname{Table of contents}
{
\hypersetup{linkcolor=}
\setcounter{tocdepth}{2}
\tableofcontents
}

\bookmarksetup{startatroot}

\chapter*{Preface}\label{preface}
\addcontentsline{toc}{chapter}{Preface}

\markboth{Preface}{Preface}

This is a Quarto book.

To learn more about Quarto books visit
\url{https://quarto.org/docs/books}.

\begin{Shaded}
\begin{Highlighting}[]
\DecValTok{1} \SpecialCharTok{+} \DecValTok{1}
\end{Highlighting}
\end{Shaded}

\begin{verbatim}
[1] 2
\end{verbatim}

\bookmarksetup{startatroot}

\chapter{Introducción}\label{introducciuxf3n}

La teoría de control óptimo trata con problemas de optimización de
sistemas dinámicos cuyo comportamiento puede verse influenciado por la
aplicación de acciones (decisiones o controles), las cuales son
seleccionadas mediante reglas conocidas como estrategias o políticas de
control. La eficiencia de cada una de tales políticas se mide mediante
un índice de funcionamiento del sistema conocido también como criterio
de optimalidad, mismo que representa, ya sea un costo o una ganancia.
Entonces el problema de control óptimo consiste en encontrar una
estrategia óptima tal que, según sea el caso, minimice o maximice un
índice de funcionamiento apropiado.

La intención de este proyecto es presentar una útil aplicación
matemática, especificamente mediante la descripción de un sistema de
inventario. La dinámica de dicho ejemplo será modelada utilizando un
Modelo de Control de Markov. De hecho, muchos trabajos de investigación
han abordado ya el estudio de los sistemas de inventario utilizando
distintos enfoques, por lo que actualmente existe ya mucha literatura al
respecto.

Para nuestro análisis consideramos espacios de estado y acciones
finitos, además de considerar una cantidad finita de transiciones para
el sistema y que todos los elementos en el MCM son ya conocidos. Esto
con el objetivo de realizar simulaciones computacionales e implementar
algoritmos ya conocidos para encontrar la función de valor óptimo, tales
como el algoritmo de la programación dinámica y el de iteración de
políticas.

\bookmarksetup{startatroot}

\chapter{Formulación del Proceso de decisión de
Markov}\label{formulaciuxf3n-del-proceso-de-decisiuxf3n-de-markov}

Considere un sistema de inventario a tiempo discreto, de un solo
producto y que cuenta con una capacidad finita \(M>0\). supongamos que
se pretende maximizar ganancias através dedecidir cuánto producto
solicitar a su proveedor a fin de solventar la demanda de sus clientes
en cada período y dependiendo del stock (producto en existencia
almacenado), cuya información se obtiene al realizar el inventario. Bajo
tal escenario, para cada \(t\in \{0,1,2, \dotso N\}\) podemos suponer
que

*\(x_t\) : representa el nivel de inventario al inicio de cada periodo
\(t\).

*\(a_n\) : representa la cantidad de producto que se ordena al inicio
del periodo \(t\).

*\(\xi\): representa la demanda del producto durante el periodo \(t\).

Se asume que la cantidad de producto que se ordena se abastece de forma
inmediata, que la demanda que no se satisface en cada periodo se pierde
y que el nivel de inventario inicial es \(x_0=x \in \mathbf{X}\).

De manera que considerando un Modelo de Control de Markov
\[( \mathbf{X}, \mathbf{A}, \{A(x): x \in \mathbf{X}\}, \mathbf{P},\mathbf{C})\]
donde, el espacio de estados y controles son:
\[\mathbf{X}=\mathbf{A}=\{0,1,2, \dotso M\}\] El conjunto de controles
admisibles cuando el nivel de inventario es \(x\in\mathbf{X}\),es
\[A(x)=\{0,1,2, \dotso,M-x\}\]

\bookmarksetup{startatroot}

\chapter{Dinámica del Modelo}\label{dinuxe1mica-del-modelo}

La dinámica del sistema evoluciona en el tiempo, de modo que en cada
etapa de desición \(t\in \{ 0,1,2, \dotso,N \}\) el nivel de inventario
es \(x_t=x \in \mathbf{X}\) y el controlador toma una desición admisible
\(a_t= a \in A(x)\). Entonces: se produce un costo \(\mathbf{C}(x,a)\);
luego, el sistema evoluciona a un nuevo estado
\(x_{t+1}=y \in \mathbf{X}\) de acuerdo a la ley de transición
\(\mathbf{P}\), la cual explicaremos a detalle en esta sección.

Consideremos que el nivel de inventario evoluciona de acuerdo a la
siguiente ecuación:
\[x_{t+1}=(x_t+a_t-\xi_t)^{+}=\max(x_t+a_t-\xi_t,0)\]

donde \(\{ \xi_t\}\) es una sucesión de variables aleatorias
independientes e idénticamente distribuidas. Dicha expresión puede ser
interpretada del siguiente modo:la cantidad de producto en el periodo
\(t+1\), será la cantidad \(x_t\) que existía hasta el periodo anterior,
más la cantidad \(a_t\) solicitada de producto, menos la demanda
\(\xi_t\) que se surte a los clientes. Dado que no hay acumulación de
demanda, la cantidad de producto no puede ser negativa, de manera que,
si no se logra surtir toda la demanda, el nivel de inventario al
siguiente periodo será cero. Suponemos que el nivel de inventario se a
venido monitoreando hata el periodp \(t\). Entonces \[
\mathbf{P}_{x,y}(a)= P[x_{t+1}=y \mid x_t=x, a_t=a]
\] \[
=P[(x_t+a_t-\xi_t)^{+}=y \mid x_t=x, a_t=a]
\] \[
=P[(x+a-\xi_t)^{+}=y]
\]

\bookmarksetup{startatroot}

\chapter{Descripción y Justificación de la Función de
Costo}\label{descripciuxf3n-y-justificaciuxf3n-de-la-funciuxf3n-de-costo}

La función de costo por etapa para cada \((x,a)\) está dada por \[
\mathbf{C}(x,a)=ca+h(x+a)+p\mathbf{E}[(\xi-x-a)^+]
\] donde \(c\) \(h\) y \(p\) son constantes positivas que representan el
costo unitario de producción, el costo unitario por almacenamiento y el
costo unitario por demanda insatisfecha, respectivamente.

\bookmarksetup{startatroot}

\chapter*{References}\label{references}
\addcontentsline{toc}{chapter}{References}

\markboth{References}{References}

\phantomsection\label{refs}
\begin{CSLReferences}{0}{1}
\end{CSLReferences}




\end{document}
